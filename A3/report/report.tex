\documentclass[11pt, a4paper]{article}

% --- PACOTES E CONFIGURAÇÕES DO DOCUMENTO ---

% Suporte para caracteres em português (acentos, ç)
\usepackage[utf8]{inputenc}
\usepackage[T1]{fontenc}
\usepackage[brazil]{babel}

% Geometria da página com margens mais agradáveis
\usepackage[a4paper, margin=2.5cm]{geometry}

% Para customização de cores
\usepackage{xcolor}

% Para incluir links clicáveis no PDF
\usepackage{hyperref}
\hypersetup{
    colorlinks=true,
    linkcolor=violet,
    filecolor=magenta,      
    urlcolor=cyan,
    pdftitle={Relatório do Projeto - Shooter Born in Heaven},
    pdfpagemode=FullScreen,
}

% Para formatação de blocos de código
\usepackage{listings}
\usepackage{courier} % Fonte monoespaçada para o código

% Define cores customizadas para os blocos de código
\definecolor{codegray}{rgb}{0.1,0.1,0.1}
\definecolor{codepurple}{rgb}{0.58,0,0.82}

% Define um estilo para blocos de código
\lstdefinestyle{mystyle}{
    backgroundcolor=\color{codegray},   
    commentstyle=\color{green},
    keywordstyle=\color{magenta},
    numberstyle=\tiny\color{gray},
    stringstyle=\color{codepurple},
    basicstyle=\footnotesize\ttfamily\color{white},
    breakatwhitespace=false,         
    breaklines=true,                 
    captionpos=b,                    
    keepspaces=true,                 
    numbers=left,                    
    numbersep=5pt,                  
    showspaces=false,                
    showstringspaces=false,
    showtabs=false,                  
    tabsize=2
}
\lstset{style=mystyle}

% Define o espaçamento entre parágrafos em vez de indentação
\usepackage{parskip}

% --- INÍCIO DO DOCUMENTO ---

\begin{document}

% --- FOLHA DE ROSTO CUSTOMIZADA ---
\begin{titlepage}
    \centering % Centraliza todo o conteúdo da página
    
    \vspace*{4cm} % Adiciona um espaço no topo da página
    
    {\Huge \bfseries Shooter Born in Heaven} % Título principal em negrito e tamanho gigante
    
    \vspace{1cm} % Espaço entre o título e o subtítulo
    
    {\Large Trabalho 03 - Programação 02} % Subtítulo em tamanho grande
    
    \vfill % Empurra o conteúdo seguinte para o meio da página
    
    % Bloco com as informações do autor
    {\large
    \begin{tabular}{l}
    \textbf{Autor:} Pietro Comin \\
    \textbf{GRR:} GRR20241955 \\
    \end{tabular}
    }
    
    \vfill % Empurra o conteúdo seguinte para a parte de baixo da página
    
    % Informações da universidade e data
    {\large
    Curso de Bacharelado em Ciência da Computação \\
    Universidade Federal do Paraná (UFPR) \\[1cm] % Espaço extra antes da data
    \today
    }
    
\end{titlepage}

% --- RESTANTE DO DOCUMENTO ---

\tableofcontents % Gera o sumário automaticamente
\newpage

\section{Descrição do Jogo}

\textbf{Shooter Born in Heaven} é um jogo de ação 2D do gênero \textit{Run 'n Gun}, inspirado em clássicos como \textit{Contra} e \textit{Metal Slug}. O jogador assume o papel de um soldado de elite que deve avançar por um cenário urbano hostil, enfrentando ondas de gangsters e culminando em uma batalha intensa contra um poderoso chefe final.

O jogo foi desenvolvido em Linguagem C, utilizando a biblioteca Allegro 5 para a renderização gráfica, gerenciamento de eventos, física e animações.

\section{Como Compilar e Executar}

O projeto utiliza um \texttt{makefile} para facilitar a compilação e a limpeza dos arquivos gerados.

\subsection*{Requisitos}
\begin{itemize}
    \item GCC (ou um compilador C compatível)
    \item Make
    \item Biblioteca Allegro 5 e seus addons (\texttt{image}, \texttt{font}, \texttt{ttf}, \texttt{primitives}) instalados.
\end{itemize}

\subsection*{Passos}
\begin{enumerate}
    \item Abra um terminal e navegue até o diretório raiz do projeto.
    
    \item Para compilar o jogo, execute o comando:
    \begin{lstlisting}[language=bash]
make
    \end{lstlisting}
    Isso irá gerar o arquivo executável \texttt{main} no diretório raiz.
    
    \item Para executar o jogo, utilize o comando:
    \begin{lstlisting}[language=bash]
./main
    \end{lstlisting}
    
    \item Para limpar os arquivos compilados (arquivos objeto e o executável), execute:
    \begin{lstlisting}[language=bash]
make clean
    \end{lstlisting}
\end{enumerate}

\section{Controles do Jogo}

\begin{table}[h!]
\centering
\begin{tabular}{|l|l|l|}
\hline
\textbf{Ação} & \textbf{Teclado Principal} & \textbf{Teclado Alternativo} \\ \hline
Movimentar & \texttt{A} / \texttt{D} & Seta Esquerda / Seta Direita \\
Pular & \texttt{W} & Seta Cima \\
Abaixar & \texttt{S} & Seta Baixo \\
Correr & \texttt{Shift Esquerdo} & \texttt{Shift Direito} \\
Atirar & \texttt{Barra de Espaço} & \\
Recarregar & \texttt{R} & \\
Pausar & \texttt{P} & \texttt{ESC} \\
Modo Debug (Hitbox) & \texttt{H} & \\ \hline
\end{tabular}
\caption{Tabela de controles do jogo}
\label{tab:controles}
\end{table}

\section{Estrutura do Projeto}
O código foi organizado de forma modular para garantir a separação de responsabilidades e facilitar a manutenção.

\begin{itemize}
    \item \textbf{\texttt{src/}}: Contém todos os arquivos de código-fonte (\texttt{.c}).
    \item \textbf{\texttt{include/}}: Contém todos os arquivos de cabeçalho (\texttt{.h}).
    \item \textbf{\texttt{assets/}}: Contém todos os recursos gráficos (sprites, backgrounds) e fontes.
    \item \textbf{\texttt{obj/}}: Diretório para os arquivos objeto gerados durante a compilação.
    \item \textbf{\texttt{makefile}}: Arquivo de compilação.
\end{itemize}

\section{Funcionalidades Implementadas}

\subsection{Requisitos Mínimos (80/80 pontos)}
\begin{itemize}
    \item[\checkmark] \textbf{Jogo Singleplayer:} O jogo é projetado para um único jogador.
    \item[\checkmark] \textbf{Menu Inicial:} Implementado com opções de "Iniciar Jogo", "Opções" e "Sair".
    \item[\checkmark] \textbf{Tela de Fim de Jogo:} Implementada, com telas distintas para vitória e derrota.
    \item[\checkmark] \textbf{Sprites do Personagem:} Todas as animações necessárias foram implementadas (parado, andando, correndo, pulando, abaixando e atirando).
    \item[\checkmark] \textbf{Mobilidade Geral:} O jogador pode andar, pular, se abaixar e atirar em todas as direções e estados.
    \item[\checkmark] \textbf{Ataque com Projétil:} O jogador e inimigos disparam projéteis com sprites próprias.
    \item[\checkmark] \textbf{Sistema de Vida:} Todos os personagens possuem um sistema de pontos de vida.
    \item[\checkmark] \textbf{Cenário com Rolagem:} O jogo utiliza um background com efeito de scroll (parallax).
    \item[\checkmark] \textbf{Inimigos e Chefe:} Implementado um inimigo normal com projéteis e um chefe final com IA própria.
    \item[\checkmark] \textbf{Sistema de Fase:} A fase é estruturada com ondas de inimigos que culminam na batalha contra o chefe.
\end{itemize}

\subsection{Funcionalidades Extras Implementadas}
\begin{itemize}
    \item[\checkmark] \textbf{[2] Inimigo Normal Extra (15/15 pontos):} Além do inimigo que atira, foi implementado um inimigo de ataque corpo a corpo (\texttt{ENEMY\_MELEE}).
    \item[\checkmark] \textbf{[3] Botão de Pausa (5/5 pontos):} O jogo pode ser pausado a qualquer momento.
    \item[\checkmark] \textbf{[4] Abaixar e Atirar Abaixado (10/10 pontos):} O jogador pode se abaixar e possui uma animação específica para atirar nesse estado.
    \item[\checkmark] \textbf{[7] Sistema de Estamina (10/10 pontos):} Implementado sistema de estamina consumido ao correr, com regeneração e HUD.
\end{itemize}

\subsection{Funcionalidades Adicionais (Não Listadas no Enunciado)}
\begin{itemize}
    \item \textbf{Sistema de Recarga de Munição:} O jogador possui munição limitada e precisa recarregar a arma (tecla \texttt{R}), adicionando uma camada tática ao combate.
    \item \textbf{Sistema de Ondas de Inimigos:} Inimigos surgem em ondas com dificuldade progressiva, criando um ritmo de combate dinâmico.
    \item \textbf{Modo "Freeplay" Pós-Vitória:} Após vencer, o jogador pode continuar na fase, onde o chefe pode reaparecer para novos desafios.
    \item \textbf{IA Avançada do Chefe:} O chefe possui múltiplos ataques, um modo "Fúria" com 50\% de vida e um sistema de "Poise" que o torna resistente a atordoamentos.
    \item \textbf{Modo de Depuração:} A tecla \texttt{H} ativa a visualização de todas as hitboxes, uma ferramenta essencial para o desenvolvimento.
\end{itemize}

\end{document}

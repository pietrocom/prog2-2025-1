\documentclass{article}
\usepackage[utf8]{inputenc}
\usepackage[brazil]{babel}
\usepackage{hyperref}

\title{Arquivador VINAc}
\author{Pietro Comin \\ GRR20241955 \\ pietro.comin@ufpr.br}
\date{\today}

\begin{document}

\maketitle

\section{Descrição do Projeto}
O VINAc é um arquivador com suporte a compressão, que permite armazenar múltiplos arquivos (membros) dentro de um único arquivo (archive) com extensão .vc. O programa oferece operações para inserção (com e sem compressão), movimentação, extração, remoção e listagem de membros.

\section{Estrutura do Projeto}
O projeto está organizado nos seguintes arquivos e diretórios:

\subsection{Diretório \texttt{include/}}
\begin{itemize}
    \item \texttt{lz/}
    \begin{itemize}
        \item \texttt{lz.h} - Cabeçalho para funções de compressão LZ
    \end{itemize}
    \item \texttt{aux.h} - Funções auxiliares
    \item \texttt{options.h} - Manipulação de opções de linha de comando
    \item \texttt{types.h} - Definições de tipos de dados
    \item \texttt{utils.h} - Utilitários diversos
    \item \texttt{vina.h} - Cabeçalho principal
\end{itemize}

\subsection{Diretório \texttt{src/}}
\begin{itemize}
    \item \texttt{lz/}
    \begin{itemize}
        \item \texttt{lz.c} - Implementação da compressão LZ
    \end{itemize}
    \item \texttt{aux.c} - Implementação de funções auxiliares
    \item \texttt{main.c} - Função principal
    \item \texttt{options.c} - Implementação das opções
    \item \texttt{utils.c} - Implementação dos utilitários
    \item \texttt{vina.c} - Implementação principal
\end{itemize}

\subsection{Outros Arquivos}
\begin{itemize}
    \item \texttt{Makefile} - Script de compilação
    \item \texttt{A1 - O Arquivador VINAc.pdf} - Especificação do projeto
    \item Arquivos de teste: \texttt{texto.txt}, \texttt{texto2.txt}, \texttt{texto3.txt}, \texttt{texto4.txt}
\end{itemize}

\section{Algoritmos e Estruturas de Dados}
\subsection{Estruturas Principais}
\begin{itemize}
    \item \texttt{struct arquivo}: Armazena metadados de cada membro
    \begin{itemize}
        \item Nome, UID, tamanho original, tamanho comprimido
        \item Ordem no archive, offset e data de modificação
    \end{itemize}
    \item \texttt{struct diretorio}: Gerencia a coleção de membros
    \begin{itemize}
        \item Quantidade de membros e vetor de ponteiros para \texttt{struct arquivo}
    \end{itemize}
\end{itemize}

\subsection{Principais Algoritmos}
\begin{itemize}
    \item \textbf{Manipulação de Arquivos}: Funções como \texttt{move()} e \texttt{move\_sequencial()} permitem reorganizar eficientemente os membros no archive, mesmo com variações de tamanho e posição.
    \item \textbf{Compressão LZ}: Implementada conforme a biblioteca fornecida, com fallback para armazenamento não-comprimido quando a compressão não for eficiente.
    \item \textbf{Gerenciamento de Memória}: Alocação dinâmica cuidadosa para evitar vazamentos, com funções dedicadas para criação e destruição de estruturas.
\end{itemize}

\section{Decisões de Implementação}
\begin{itemize}
    \item \textbf{Modularização}: O código foi dividido em funções pequenas e especializadas para facilitar depuração e manutenção.
    \item \textbf{Manipulação Direta em Disco}: Para evitar uso excessivo de memória, os dados dos membros são manipulados diretamente no arquivo.
    \item \textbf{Atualização de Metadados}: Funções como \texttt{atualiza\_metadados()} garantem a consistência das informações após operações.
\end{itemize}

\section{Dificuldades e Soluções}
\begin{itemize}
    \item \textbf{Manipulação de Arquivos Variáveis}: A principal dificuldade foi lidar com membros de tamanhos diferentes e suas posições no archive. As funções \texttt{move()} e \texttt{move\_sequencial()} resolvem este problema movendo blocos de dados de forma segura.
    \item \textbf{Compressão Ineficiente}: Quando a compressão resulta em arquivo maior que o original, o programa automaticamente armazena os dados descomprimidos.
    \item \textbf{Valgrind Warnings}: O erro reportado pelo Valgrind (uninitialized bytes) foi considerado de baixo impacto, pois ocorre em operações de buffer interno da biblioteca padrão.
\end{itemize}

\section{Bugs Conhecidos}
\begin{itemize}
    \item O Valgrind reporta acesso a bytes não inicializados durante operações de escrita, mas este comportamento parece ser inócuo e relacionado ao buffer interno da biblioteca padrão.
\end{itemize}

\section{Compilação e Uso}
\subsection{Compilação}
Execute \texttt{make} para compilar o programa, que será gerado em \texttt{login/vinac}.

\subsection{Exemplos de Uso}
\begin{verbatim}
./login/vinac -ip teste.vc texto.txt texto2.txt texto3.txt texto4.txt
./login/vinac -m teste.vc texto.txt texto2.txt
./login/vinac -ic teste.vc texto2.txt
./login/vinac -x teste.vc
\end{verbatim}

\section{Makefile}
O Makefile fornecido:
\begin{itemize}
    \item Compila cada módulo separadamente
    \item Gera o executável em \texttt{login/vinac}
    \item Suporta \texttt{make clean} para remover arquivos temporários
    \item Inclui dependências entre arquivos
\end{itemize}

\end{document}